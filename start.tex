%1
\section{はじめに}
\label{sec:start}

人は五感によってさまざまな刺激を外部から受けている.人の知覚というのはこれらの感覚が相互作用することにより形成されることが知られている.
「味」というのは味覚だけではなくさまざまな感覚が統合されている.今回はその中でも味覚に着目し,それらと密接に関係のある嗅覚と視覚からアプローチを行っていく.

人の知覚の8割は視覚と言われており、情報のほとんどを視覚に頼っている.皿にのせられた彩りからおいしさを感じるように,色覚からの情報は重要なファクターとなっている.

嗅覚は、味覚においての8割以上を占めていると言われている.嫌なものを食べるときには鼻をつまんでニオイを分からなくするようにするという経験は分かりやすい例である.これらは日々の体験からもよくわかる.


これらを如実に表しているものを例に挙げるとするならばかき氷である.
かき氷は,氷を削ったり砕いたりしたものに,シロップをかけたものである.
その味の決め手となっているシロップは,果汁に代わる甘味料に加えて,香料や着色料を加えて作るものであるが,かき氷の味は結局のところはすべて同じ味であるというのはよく聞く話である.
現にかき氷のシロップの原料は,どの味においても着色料と香料の種類の違いだけであった.
そうであるにもかかわらず人がかき氷の味を区別するのは,脳がかき氷の見た目とニオイで錯覚を起こし,勝手にイメージで味を作っているからである.
このことから着色料と香料を別のもので補うことができればさまざまな味のかき氷を再現することができると考えた.

本稿では,着色料と香料の代わりとして,視覚においてはLED光源をを使用して着色料と同等の色を再現し,嗅覚においては,アロマオイル等を使用し鼻から得られるものを用意する.
これによって被験者が味にどのような差異を調査し,研究していく.
